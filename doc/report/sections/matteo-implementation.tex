\subsection{Matteo Brocca}
Durante il corso del progetto mi sono occupato di studiare ed implementare:
\begin{itemize}
	\item Rappresentazione grafica degli Anythings tramite Asset ed animazioni
    \item Definizione del FireModel e ShotModel per la gestione dei proiettili
    \item Gestione delle statistiche degli Anythings
    \item Character controllato dal giocatore
    \item Boss, PrologEnemyModel e definizione del suo comportamento tramite Prolog
    \item Item che il Character può raccogliere per modificare le proprie statistiche
    \item HUD per la visualizzazione delle statistiche a schermo
\end{itemize}

\subsubsection{Rappresentazione grafica degli Anythings tramite Asset ed animazioni}
È stata creata una libreria per gli \textbf{AnythingAsset}, 
la quale codifica le informazioni di base necessare a disegnare a schermo un certo Asset ed implementa metodi condivisi. 

Questa organizzazione ci ha permesso di generare velocemente le specifiche View per Character, Enemies, Bosses, Items and Elements.

Facciamo un esempio:
Il CharacterModel, viene rappresentato a schermo dalla sua relativa CharacterView e CharacterViewModel 
che estendono relativamente da AnythingView ed AnythingViewModel

La CharacterView definisce solo la funzione di "disegno" che sfrutta il metodo drawComponents di AnythingAsset 
e viene mixata con il tipo di carattere che si vuole disegnare a schermo, in questo caso Isaac.
Il trait Isaac estende IsaacAsset per definire come disegnare ogni singolo componente a schermo.
A sua volta il trait IsaacAsset è quello che estende il trait base AnythingAsset 
e definisce le dimensioni e nome della Sprite che viene caricata durante l'avvio del gioco.

Nel momento in cui si decida di modificare l'estetica del Character (il \textit{"cosa"}), da "Isaac" a "Mario",
senza dover modificare il \textit{"come"} questo viene rappresentato (da una testa, un corpo, un ombra, ...),
basterà create un nuovo trait Mario e relativo MarioAsset.

\begin{lstlisting}[language=Scala]
trait AnythingAsset {
	...
	protected def drawComponents(components: List[SceneNode]): Group = ...
}

trait IsaacAsset extends AnythingAsset {
  	...
}

trait Isaac extends IsaacAsset{
	...
	def headView(model: CharacterModel, viewModel: CharacterViewModel): Graphic[Material.Bitmap] = {...}
	def bodyView(model: CharacterModel): Sprite[Material.Bitmap] = = {...}
}

trait CharacterView[VM <: AnythingViewModel[CharacterModel] | Unit] extends AnythingView[CharacterModel, VM] {}

object CharacterView extends CharacterView[CharacterViewModel] with Isaac {
	...
	def view(contex: FrameContext[StartupData], model: Model, viewModel: ViewModel): View =
		...
		drawComponents(List(shadowView, bodyView(model), headView(model, viewModel)))
}
\end{lstlisting}	

\subsubsection{Definizione del FireModel e ShotModel per la gestione dei proiettili}
Ogni proiettile generato dal FireModel è uno ShotModel.

La factory che ne permette la creazione durante la generazione dell'evento ShotEvent 
richiede anche quale sia la relativa View per la rappresentazione grafica del proiettile.
Si può infatti notare che il FireModel richiede la definizione della funzione di creazione di questa View di tipo ShotView (pattern strategy: factory as a function), un trait che sfrutta il \textbf{Self-Type} per richiedere espressamente di essere mixato 
con il trait \textbf{ShotAsset} che definisce le informazioni di base che ogni proiettile deve avere ed implementa la funzione \textbf{drawShot}

\begin{lstlisting}[language=Scala]
trait FireModel extends AnythingModel with StatsModel {
  	type Model >: this.type <: FireModel

	...

  	val shotView: () => ShotView[_]

  	...
}

class SingleShotView extends ShotView[Unit] with SimpleAnythingView {
  this: ShotAsset =>
  type View = Group

  def view(contex: FrameContext[StartupData], model: Model, viewModel: ViewModel): View =
    drawComponents(List(drawShot))

  ...
}
\end{lstlisting}

Un esempio dell'utilizzo del FireModel si può apprezzare all'interno del CharacterModel 
dove viene definita la shotView come un SingleShotView con il trait ShotBlue
e viene implementato il computeFire con il mapping dei tasti premuti dall'utente.

\begin{lstlisting}[language=Scala]
case class CharacterModel(...) extends AliveModel
		with DynamicModel
		with FireModel
		with DamageModel
		with SolidModel {

	val shotView        = () => new SingleShotView() with ShotBlue
	
	...
}
\end{lstlisting}

\subsubsection{Gestione delle statistiche degli Anything}
Si è realizzato il trait \textbf{StatsModel} da mixare con un qualsiasi Anything 
per gestire l'aggiornamento di queste caratteristiche nell'ambito dell'immutabilità.

Il metodo "sumStat" è l'unico utilizzato realmente all'interno del gioco e permette di 
aggiornare il valore di una determinata caratteristica sommando il valore proveniente da un Item raccolto 
verificando in automatico che il valore non diventi negativo.

Gli altri metodi "changeStats" e "changeStat" sono solo stati previsti per eventuali sviluppi futuri.  

\begin{lstlisting}[language=Scala]
trait StatsModel {
  type Model >: this.type <: StatsModel

  val stats: Stats

  def withStats(stats: Stats): Model

  def changeStats(context: FrameContext[StartupData], newStats: Stats): Outcome[Model] = Outcome(withStats(newStats))

  def changeStat(context: FrameContext[StartupData], property: StatProperty): Outcome[Model] =
    Outcome(withStats(stats + property))

  def sumStat(context: FrameContext[StartupData], property: StatProperty): Outcome[Model] =
    Outcome(withStats(stats +++ property))
}
\end{lstlisting}

Questo trait usufruisce della libreria \textbf{Stats} la quale è implementata seguendo il pattern \textbf{Pimp My Library} 
per mettere a disposizione i propri metodi all'interno dello specifico dominio applicativo (\textbf{DSL}).

La libreria definisce che cosa sono le Stats, una mappa di proprietà da nome a valore.
Sono stati implementate delle \textbf{Conversion} per utilizzare i valori definiti come Double anche in situazioni 
dove si accettano Int o String.
Inoltre sono state realizzate delle \textbf{extension} per definire operatori specifici per manipolarle.

\begin{lstlisting}[language=Scala]
package lns.scenes.game.stats

enum PropertyName:
	case 	MaxLife, Invincibility, MaxSpeed, Range, KeepAwayMin, KeepAwayMax, Damage, 
		FireDamage, FireRange, FireRate,FireSpeed

type PropertyValue = Double
type StatProperty  = (PropertyName, PropertyValue)
type Stats         = Map[PropertyName, PropertyValue]

given Conversion[PropertyValue, Int] with
	def apply(v: PropertyValue): Int = v.toInt

given Conversion[PropertyValue, String] with
	def apply(v: PropertyValue): String = v.toString

extension (p: PropertyValue) {
	def |+|(v: PropertyValue): PropertyValue = p match {
	case p if (v + p) < 0 => 0
	case _                => BigDecimal(v + p).setScale(2, BigDecimal.RoundingMode.HALF_UP).toDouble
	}
}

extension (p: PropertyName) {
	def @@(s: Stats): PropertyValue = s.getOrElse(p, 0.0)
}

extension (stats: Stats) {
	def +++(p: StatProperty): Stats =
	stats + stats.get(p._1).map(x => p._1 -> (x |+| p._2)).getOrElse(p)
}
\end{lstlisting}

\subsubsection{Boss, PrologEnemyModel e definizione del suo comportamento tramite Prolog}
Per la creazione del boss, il cui comportamento è definito tramite linguaggio Prolog, è stato creato il trait \textbf{PrologEnemyModel}. 
Grazie a Scala 3, il trait accetta il paramentro "\textit{name}" che definisce il nome del file prolog da interrogare e che sarà stato opportunamente caricato tra gli asset di gioco.

Se il suo EnemyStatus attuale (tipico degli EnemyModel) è di tipo \textit{Idle} e non è definita nessuna \textit{EnemyAction}, 
allora viene interrogato il codice Prolog. 
Il \textbf{goal} che viene passato è una stringa prodotta dal metodo del quale si richiede l'override.
La risposta del prolog, catturata dal gameLoop, esegue il metodo \textbf{behaviour} del chiamante implementato dalla classe che lo estende. 

\begin{lstlisting}[language=Scala]
trait PrologEnemyModel(name: String) extends EnemyModel {
  type Model >: this.type <: PrologModel

  val prologClient: PrologClient

  def withProlog(prologClient: PrologClient): Model

  protected def goal(context: FrameContext[StartupData])(gameContext: GameContext): String

  def behaviour(response: Substitution): Outcome[Model]

  protected def consult(context: FrameContext[StartupData])(gameContext: GameContext): Outcome[Model] = ...

  override def update(context: FrameContext[StartupData])(gameContext: GameContext): Outcome[Model] = ...
}

case class BossModel(...) ) 
	extends PrologEnemyModel("loki")
	with FireModel
	with Traveller {

	...

	protected def goal(context: FrameContext[StartupData])(gameContext: GameContext): String = ...

	def behaviour(response: Substitution): Outcome[Model] = ...
}
\end{lstlisting}

Il boss di nome "Loki" che è stato implementato può eseguire 5 diverse azioni, 3 di attacco, 1 di movimento ed 1 di difesa,
che hanno una probabilità in base allo stato di vita del Boss e del Character. 
Viene definito \textit{low} il livello di vita se questo è inferiore al 50\% della vita massima, altrimenti \textit{high}

\begin{lstlisting}[language=Scala]
% estratto del file prolog "assets/prolog/loki.pl"
... 
% List of probability value for actions based on Boss and Character life level
% @high|low (Boss)
% @high|low (Character)
% -Probability List[Attack, Defense, Move]
getActionsProbability(high, high, [0.7, 0.0, 1.0]).
getActionsProbability(high, low,  [0.5, 0.0, 1.0]).
getActionsProbability(low,  high, [0.7, 1.0, 0.0]).
getActionsProbability(low,  low,  [0.5, 0.7, 1.0]).
...
\end{lstlisting}

Di seguito vengono descritte nel dettaglio le singole azioni:
\begin{itemize}
	\item attack1(direction): viene generato un proiettile nella direzione del Character se questo si trova sullo stesso asse x o y.
	\item attack2: vengono generati 4 proiettili in contemporanea lungo i 2 assi x, y
	\item attack3: vengono generati 4 proiettili in contemporanea in diagonale lungo i 2 assi x, y
	\item move(x,y): il boss sposta rapidamente verso il punto x, y occupato dal Character
	\item defence(x,y): il boss si teletrasporta nel punto x,y se questo non è occupato da una roccia, altrimenti viene cercato il primo posto disponibile nel suo intorno
\end{itemize}