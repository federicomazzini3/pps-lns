\subsection{Alan Mancini}
Durante il corso del progetto mi sono occupato di studiare ed implementare:
\begin{itemize}
    \item AnythingModel con alcuni suoi sottotipi come DynamicModel e AliveModel e sperimentazione del workflow per consentire mixin/estensione e l'update del Model immutabile 
    \item Macro per ridurre il boilerplate code quando si estende un AnythingModel
    \item Utilities per semplificare l'update di un insieme di AnythingModel
    \item AnythingView e AnythingViewModel di base e sperimentazione del workflow derivante
    \item Nemici per quanto concerne Model ed i loro comportamenti riutilizzabili
    \item Servizio Prolog come sottosistema di Indigo, integrazione di TauProlog e intefacciamento grazie a Scala.js, ed infine client per consultazioni da parte del gioco
    \item Generazione del dungeon randomica con Prolog
    \item Visualizzazione mini-mappa del dungeon
\end{itemize}
\subsubsection{AnythingModel e immutabilità}
Per quanto riguarda il trait AnythingModel e suoi trait sottotipi/mixins il problema è quello di consentire l'update dello stesso nel contesto dell'immutabilità ritornando quindi una \textbf{copia aggiornata di se stesso e del tipo corrente}. 
Scala infatti attualmente non dispone di un sistema per copiare un oggetto aggiornando solo alcune sue proprietà, se non riapplicando il costruttore creando un nuovo oggetto o utilizzando il metodo copy di una case class, ma a partire da un trait base non si conosce ancora quale classe/case class lo estenderà.
Una tecnica che potrebbe sembrare idonea è l'utilizzo di un self-type richiedendo un tipo anonimo che implementi un metodo copy, ma occorrebbe specificare tutti i suoi argomenti e questi non si conoscono in quanto dipendono dalla classe finale e come questa viene mixata.

Sebbene una soluzione consigliata è quella di impiegare polimorfismo ad-hoc con typeclass, abbiamo riscontrato che nel nostro caso volendo mantenere il mixing degli oggetti questo pattern non è facilmente applicabile.

La soluzione trovata è quindi quella di applicare il pattern \textbf{F-Bounded Polimorphism} con type-member.

\begin{lstlisting}[basicstyle=\tiny]
trait AnythingModel {
    type Model >: this.type <: AnythingModel
    
    val id: AnythingId
    val view: () => AnythingView[Model, _]

    ...

    def update(context: FrameContext[StartupData])(gameContext: GameContext): Outcome[Model] =
        Outcome(this)
    }
}  
\end{lstlisting}

\begin{lstlisting}[basicstyle=\tiny]
trait DynamicModel extends AnythingModel with StatsModel {
  type Model >: this.type <: DynamicModel

  val speed: Vector2
  val collisionDetected: Boolean

  def withDynamic(boundingBox: BoundingBox, speed: Vector2, collisionDetected: Boolean): Model

  override def update(context: FrameContext[StartupData])(gameContext: GameContext): Outcome[Model] =
    for {
      superObj <- super.update(context)(gameContext)
      (newSpeed, newPosition) = computeMove(context)(gameContext)
      boundLocation           = gameContext.room.boundPosition(newPosition)
      newObj = superObj
        .withDynamic(boundLocation, newSpeed, boundLocation.position != newPosition.position)
        .asInstanceOf[Model]
    } yield newObj

} 
\end{lstlisting}

Da notare la necessità di un cast a Model in quanto il compilatore Scala non riconosce che il type member di ritorno è lo stesso definito poco sopra: questo cast comunque è legittimo e sicuro. 

Per concludere la descrizione di Anything: ogni Model viene creato specificando la sua View (factories as functions in linea con il \textbf{pattern strategy}) in modo tale che, quando una collezione di oggetti viene renderizzata, non è necessario fare type-match e indicare la corretta View da utilizzare in base al Model.


\subsubsection{Macro per ridurre boilerplate code}
Quando si definisce il Model di un oggetto di gioco che mixa diversi comportamenti è necessario implementare altrettanti metodi del tipo with-Comportamento in modo da permettere la sua copia aggiornata. 

\begin{lstlisting}[basicstyle=\tiny]
def withDynamic(x,y,z) = copy(x=x,y=y,z=z); 
\end{lstlisting}

Per questo ho cercato un modo per generare in automatico questi metodi per una case class qualsiasi e l'unico sistema era scrivere una macro attivata da una annotation, ma Scala 3 al momento non supporta questa soluzione possibile su Scala 2.

Quindi ho implementato la \textbf{copyMacro} che consente di eseguire il metodo copy con come argomenti quelli dello scope dove viene attivata la macro, sfruttando le potenzialita della ancora non documentata Scala Reflection API. 

\begin{lstlisting}[basicstyle=\tiny]
def withDynamic(x,y,z) = copyMacro
\end{lstlisting}

\subsubsection{Utilities per update di collection di Anything}

Quando si esegue l'update di una collezione di oggetti AnythingModel o AnythingViewModel ad esempio Map[AnythingId, AnythingModel] si ottiene una Map[AnythingId, Outcome[AnythingModel]] ma quello che occorre è ottenere la Map originale unendo le Outcome e gli eventi contenuti da ciascuna.

\begin{lstlisting}[basicstyle=\tiny]
anythings.map((id, any) => id -> any.update(context)(GameContext(this, character)))
\end{lstlisting}

Per comodità ho applicato il \textbf{pattern adapter} fornendo un apposito convertitore implicito, di seguito quello per AnythingViewModel.

\begin{lstlisting}[basicstyle=\tiny]
given Conversion[Map[AnythingId, Outcome[AnythingViewModel[_]]], Outcome[Map[AnythingId, AnythingViewModel[_]]]] with
  def apply(set: Map[AnythingId, Outcome[AnythingViewModel[_]]]): Outcome[Map[AnythingId, AnythingViewModel[_]]] =
    set.foldLeft(Outcome(Map[AnythingId, AnythingViewModel[_]]().empty))((acc, el) =>
      acc.merge[AnythingViewModel[_], Map[AnythingId, AnythingViewModel[_]]](el._2)((set, el2) => set + (el._1 -> el2))
    )
\end{lstlisting}