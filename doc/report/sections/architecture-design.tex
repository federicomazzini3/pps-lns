\section{Design architetturale}
%Design architetturale (architettura complessiva, descrizione di pattern architetturali usati, componenti del sistema distribuito, scelte tecnologiche cruciali ai fini architetturali -- corredato da pochi ma efficaci diagrammi)

Di seguito vengono analizzate l'architettura complessiva dell'applicazione e le decisioni critiche riguardanti pattern architetturali e tecnologie rilevanti.
Prima di pensare a qualsiasi architettura per il gioco da realizzare, ciò che si è cercato di fare è stato individuare l'engine ideale per il progetto. 
La scelta è ricaduta su Indigo in quanto: 
\begin{itemize}
    \item E' un framework orientato al game development, strutturato in modo da favorire l'implementazione di componenti riusabili. 
    \item E' scritto in Scala e favorisce lo sviluppo tramite paradigma funzionale.
    \item Permette il gioco su browser, come da requisito di business, mediante compilazione tramite Scala.js.
\end{itemize}
I concetti chiave di Indigo che ci guidano nello sviluppo del gioco sono 
\begin{itemize}
    \item Model, View, ViewModel
    \item Game loop con modello ad eventi
    \item Scene
\end{itemize}

\subsection{Model, View, ViewModel}
Il pattern architetturale \textbf{MVVM} (Model, View, ViewModel) è una variante del famoso \textbf{MVC} (Model, View, Controller). 
Scopo di questo pattern è quello di separare il modello dei dati dalla sua rappresentazione, tenendo entrambi puri e funzionali al loro scopo, frapponendo tra loro un elemento "ibrido", il ViewModel.

\paragraph{Model}
Rappresenta il modello di gioco puro, indipendentemente dalla sua rappresentazione visuale. 
Esso contiene lo stato del gioco e la sua logica. 

\paragraph{ViewModel}
Si interpone tra Model e View. Ha lo scopo di mantenere alcuni dati utili per la rappresentazione che tuttavia non concernono la logica di gioco, devono perciò rimanere separati e non essere inclusi all'interno del Model.

\paragraph{View}
Si occupa di offrire una rappresentazione del Model e del ViewModel a schermo. Mantiene quindi la logica di rappresentazione di questi, senza contenere però alcuna logica di gioco.

\subsection{Game loop con modello ad eventi}
La logica di Indigo si basa su un game loop che processa, ad ogni iterazione, una coda di eventi. 
Ad ogni ciclo, il framework esegue le seguenti operazioni:
\begin{itemize}
    \item Per ogni evento, update del Model
    \item Per ogni evento, update del ViewModel
    \item Presentazione e rendering degli elementi a video
    \item Reset della coda di eventi
\end{itemize}

Il concetto di immutabilità, presente nel paradigma funzionale, qui si traduce non in un update di uno stato interno del modello, ma bensì nella generazione di un nuovo modello aggiornato. 

\subsection{Scene}
Le scene sono un modo di organizzare il codice secondo una logica di gioco ben definita. Sono un meccanismo di suddivisione che permette di individuare delle "fasi" di gioco da sviluppare in modo separato le une dalle altre.
In ogni istante all'interno del gioco, il modello, l'aggiornamento e la presentazione sono delegate alla scena corrente in esecuzione, la quale si occupa anche di gestire gli eventi. 

Abbiamo identificato quattro scene principali su cui andare a definire Lost 'n Souls. 

