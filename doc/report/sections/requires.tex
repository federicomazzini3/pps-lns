\section{Requisiti}

\subsection{Requisiti di business}
%Requirement di alto livello stabiliscono perché si sta facendo 
%il sistema e quali siano i suoi vantaggi. 
\begin{itemize}
    \item Creazione di un gioco di genere Roguelike single player con la possibilità per un utente di comandare un personaggio all'interno di una mappa composta da stanze, affrontare nemici, raccogliere oggetti e sconfiggere un nemico finale.
    \item Possibilità di gioco su browser
\end{itemize}

\subsection{Requisiti utente}
%Quando un utente usa il sistema, cosa vuole fare e cosa si aspetta?
%Raccolgono le aspettative per un utente.
%Un modo tipico per documentare i requisiti utente sono le user stories.

Matteo Brocca in questo progetto ricopre il ruolo di esperto di dominio e committente. 
E' un appassionato di giochi Roguelike ma essendo troppo bravo li ha finiti tutti. 
Da qui l'idea di crearne uno nuovo per lui. 

\begin{enumerate}
    \item L'utente potrà 
    \begin{enumerate}
        \item avviare una nuova partita da un menu contestuale
        \item controllare con la tastiera il proprio personaggio all'interno della stanza per
        \begin{enumerate}
            \item spostarsi e cambiare direzione
            \item sparare ai nemici che possono essere di diversa tipologia
            \item evitare elementi bloccanti o di disturbo se presenti
            \item raccogliere oggetti utili all'aumento delle sue caratteristiche
            \item spostarsi da una stanza all'altra attraverso delle porte
        \end{enumerate}
        \item visualizzare in modo continuo le sue statistiche e i suoi punti vita
        \item visualizzare in modo continuo una mappa del dungeon che indichi la sua posizione corrente
       
    \end{enumerate}
    \item L'utente dovrà riuscire a 
    \begin{enumerate}
        \item distinguere visivamente nemici, oggetti e elementi di disturbo
        \item capire in che direzione si sta muovendo
        \item capire dove sta sparando
        \item capire di aver colpito un nemico
        \item capire di aver raccolto un oggetto
        \item capire qual'è la stanza del boss
        \item capire che sta fronteggiando un boss
        \item capire di aver vinto o perso una partita
    \end{enumerate}
\end{enumerate}

\subsection{Requisiti funzionali}
%Funzionali: statement dettagliati delle funzionalità del sistema indicate
%in modo chiaro (organizzarli permette uno sviluppo del progetto rigoroso)
\begin{enumerate}
    \item Menù di gioco
    \begin{enumerate}
        \item presenza di un tasto per avviare una nuova partita
    \end{enumerate}
    \item Caricamento del gioco
    \begin{enumerate}
        \item presenza di una schermata durante il caricamento degli elementi di gioco
    \end{enumerate}
    \item Generazione del dungeon/mappa di gioco 2D in maniera casuale
    \begin{enumerate}
        \item Durante la generazione deve essere visualizzata una schermata di attesa
        \item Un dungeon è formato da più stanze quadrate
        \item Ogni stanza è fisicamente adiacente ad almeno un'altra stanza
        \item Tra stanze adiacenti è presente una porta che le collega
        \item La disposione delle stanze avviene favorendo forme di mappa complesse simil labirinto 
        \item Una stanza è di una tipologia fra
            \begin{enumerate}
                \item Vuota
                    \begin{enumerate}
                        \item Stanza completamente vuota
                        \item La partita comincia con il personaggio in una stanza vuota
                        \item In un dungeon ci sono circa un 10\% di stanze vuote
                    \end{enumerate}
                \item Oggetto
                    \begin{enumerate}
                        \item Contiene al centro un singolo oggetto scelto randomicamente
                        \item In un dungeon ci sono circa un 15\% di stanze oggetto
                    \end{enumerate}
                \item Combattimento
                    \begin{enumerate}
                        \item Il contenuto è generato randomicamente
                        \item Contiene alcuni nemici
                        \item Contiene alcuni elementi bloccanti disposti in modo da non ostruire l'accesso alle porte da parte del personaggio
                        \item Le porte si chiudono quando il giocatore entra nella stanza
                        \item Le porte si aprono quando il giocatore ha eliminato tutti i nemici
                        \item In un dungeon ci sono circa un 75\% di stanze combattimento
                    \end{enumerate}
                \item Boss
                    \begin{enumerate}
                        \item Contiene solamente il nemico boss
                        \item Una sola nel dungeon
                        \item Adiacente ad una ed una sola altra stanza
                    \end{enumerate}
            \end{enumerate}
        
        
    \end{enumerate}

    \item Scena di gioco
    \begin{enumerate}
        \item A video, dopo la generazione del dungeon, è visibile
        \begin{enumerate}
            \item al centro dello schermo e rappresentata con vista dall'alto solo la stanza dove è correntemente presente il personaggio
            \item a sinistra della stanza l'elenco delle statistiche del personaggio
            \item a sinistra della stanza una mini mappa del dungeon
            \begin{enumerate}
                \item Le stanze vuote e combattimento sono di colore neutro (bianco)
                \item La stanza corrente, quelle oggetto e la stanza boss sono evidenziate con colori propri 
            \end{enumerate}
        \end{enumerate}
        \item All'interno di una stanza nemici, oggetti, elementi bloccanti e personaggio, in linea generale
            \begin{enumerate}
                \item sono vincolati a stare nei limiti del perimetro rappresentato dal pavimento
                \item non possono fisicamente attraversarsi tra loro sovrapponendosi graficamente, tuttavia la parte in alto di un'entità può sovrapporsi alla parte bassa di un'altra senza generare collisione ad emulare altezze diverse
            \end{enumerate}
        \item All'interno di una stanza un colpo sparato
        \begin{enumerate}
            \item è rappresentato con un cerchio colorato
            \item è vincolato a stare nei limiti del perimetro rappresentato dal pavimento
            \item colpisce un'entità qualsiasi non appena il primo entra in contatto con il rettangolo che graficamente racchiude interamente la seconda 
            \item quando colpisce qualcosa il cerchio viene sostituito da un esplosione animata
        \end{enumerate}
        
    \end{enumerate}
    
    \item Personaggio controllabile con caratteristiche e punti vita
    \begin{enumerate}
        \item Il personaggio è caratterizzato da
            \begin{enumerate}
                \item Punti vita
                \item Tempo invulnerabilità dopo essere stato colpito
                \item Velocità di movimento
                \item Rate dello sparo
                \item Danno dei proiettili
                \item Velocità dei proiettili
                \item Range dei proiettili
            \end{enumerate}
        \item Il personaggio è controllato dall'utente con la tastiera
        \begin{enumerate}
            \item L'utente può muovere il personaggio nelle quattro direzioni principali (sopra, sotto, destra, sinistra)
            \item L'utente può sparare uno o più colpi verso una delle quattro direzioni principali (sopra, sotto, destra, sinistra)
            \item Il personaggio può uscire dalla stanza attraverso le porte
        \end{enumerate}
        \item Il personaggio è visivamente costituito da
        \begin{enumerate}
            \item Una testa orientata in modo da rappresentare la direzione di fuoco, se il personaggio non sta sparando allora è orientata nella direzione di movimento
            \item Un corpo orientato ed animato in modo da rappresentare la direzione di movimento
        \end{enumerate}
        \item Il personaggio infligge danno ai nemici quando un suo colpo colpisce un nemico
        \item Il personaggio può raccogliere oggetti che modificano le sue caratteristiche
        \item Il personaggio muore e la partita termina mostrando un messaggio "game over" quando la sua vita arriva a zero
    \end{enumerate}
    
    \item Nemici con diverse caratteristiche e comportamenti
    \begin{enumerate}
        \item Un nemico può essere nello stato
        \begin{enumerate}
            \item Inattivo: non esegue azioni
            \item Attacco: esegue azioni di attacco nei confronti del personaggio
            \item Difesa: esegue azioni per difendersi dal personaggio
            \item Nascosto: esegue azioni per nascondersi dal personaggio
        \end{enumerate}
        \item Ogni nemico è caratterizzato da
            \begin{enumerate}
                \item Punti vita
                \item Velocità di movimento
                \item Danno da contatto
                \item Se il nemico spara
                \begin{enumerate}
					\item Rate dello sparo
					\item Danno dei proiettili
					\item Velocità dei proiettili
					\item Range dei proiettili
                \end{enumerate}
            \end{enumerate}
        \item Un nemico infligge danno al personaggio quando vi entra in collisione
        \item Un nemico che spara infligge danno al personaggio quando un suo colpo vi entra in collisione
        \item Un nemico se si muove lo fa solo all'interno di una sola stanza
        \item Un nemico muore quando la sua vita arriva a zero, al suo posto viene visualizzata un esplosione animata
        \item Quattro tipologie di nemico
            \begin{enumerate}
                \item Nerve
                    \begin{enumerate}
                        \item Sta sempre nello stato di attacco
                        \item Sta immobile
                        \item Costituito da un corpo
                    \end{enumerate}
                \item Boney
                    \begin{enumerate}
                        \item Sta sempre nello stato di attacco
                        \item Si sposta continuamente in direzione del personaggio
                        \item Costituito da testa e corpo orientati nella direzione di movimento
                    \end{enumerate}
                \item Mask
                    \begin{enumerate}
                        \item Sta sempre nello stato di attacco
                        \item Si sposta mantenendo una distanza predefinita dal personaggio
                        \item Spara colpi in direzione del personaggio
                        \item Costituito da una testa orientata nella direzione di fuoco
                    \end{enumerate}
                \item Parabite
                    \begin{enumerate}
                        \item Sta 1 secondo nello stato "inattivo", poi passa allo stato di "attacco", poi allo stato "nascosto" per 2 secondi poi ricomincia da capo
                        \item Quando passa da inattivo ad attacco calcola una linea retta in direzione del personaggio e la percorre velocemente
                        \item Quando passa allo stato "nascosto" può essere attraversato da personaggio o altri nemici e non può essere colpito
                        \item Costituito da un corpo orientato nella direzione di movimento
                    \end{enumerate}
            \end{enumerate}
        \item Boss
        \begin{enumerate}
            \item E' un nemico speciale che se viene sconfitto la partita termina visualizzando al giocatore il messaggio "hai vinto"
            \item Il suo comportamento è determinato da 5 diverse azioni scelte casualmente con una certa probabilità in base al suo stato di vita e quello del personaggio.
            \begin{enumerate}
				\item Attacco con singolo proiettile in direzione del personaggio se questo si trova sullo stesso asse x o y all'interno della stanza
				\item Attacco che spara 4 proiettili in contemporanea: sopra, sotto, destra e sinistra
				\item Attacco che spara 4 proiettili in contemporanea in diagonale: sopra, sotto, destra e sinistra
				\item Si sposta verso il personaggio per aggredirlo
				\item Si teletrasporta in un punto della stanza specchiato rispetto a quello attuale. Se il punto è occupato da un elemento bloccante, trova il primo posto libero nel suo intorno
			\end{enumerate}
            \item Costituito da un corpo
        \end{enumerate}
    \end{enumerate}
    
    \item 10 Oggetti che il personaggio può raccogliere per incrementare le proprie caratteristiche
    \begin{enumerate}
        \item Arrow: incermenta il rate dello sparo 
        \item Drop: incermenta il danno dei proiettili
        \item Eye: incrementa il danno dei proiettili ed il rate dello sparo
        \item Fireball: incrementa il danno e la velocità dei proiettili
        \item Glasses: incrementa il range dei proiettili
        \item Heart: incrementa la vita massima di 1
        \item Juice: incrementa la velocità di spostamento e danno del fuoco
        \item Mushroom: incrementa la vita massima di 2
		\item Syringe: incrementa la velocità di spostamento, il rate dell sparo e la velocità dei proiettili
		\item Tail: incrementa il danno ed il range dei proiettili
    \end{enumerate}
         
\end{enumerate}


\subsection{Requisiti non funzionali}
%Non funzionali: statement dettagliati che raccolgono la qualità del 
%comportamento o sui vincoli del sistema

\begin{enumerate}
    \item Fluidità di gioco
    \begin{enumerate}
        \item Deve essere garantita una velocita di almeno 30 FPS su processore Apple M1 in situazioni di gioco "caotiche", dove per caotico si intende la presenza all'interno di una stanza del Character comandato dal giocatore, 5 nemici, 10 shot e 10 elementi bloccanti.
    \end{enumerate}
\end{enumerate}


\subsection{Requisiti di implementazione}
%Implementazione: requirement posti agli sviluppatori. Danno indicazioni 
%sul lavoro da fare. Possono comprendere: tipi di strumenti utilizzati, 
%modi di sviluppo, documentazione fornita.

\begin{enumerate}
\item Implementazione mediante Scala 3
\item Implementazione mediante Prolog della generazione del dungeon, delle stanze combattimento e del comportamento del Boss
\item Testing mediante ScalaTest
\item Sviluppo del software in modo da considerare possibili future estensioni di gioco come:
\begin{enumerate}
    \item Aggiunta di diversi personaggi con cui giocatore
    \item Aggiunta di nuovi nemici e boss
    \item Aggiunta di nuovi elementi bloccanti
    \item Aggiunta di nuovi oggetti con diverse caratteristiche
\end{enumerate}
\item Rilascio del gioco su server tramite GitHub Actions
\end{enumerate}