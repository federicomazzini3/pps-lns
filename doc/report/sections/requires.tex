\section{Requisiti}

\subsection{Requisiti di business}
%Requirement di alto livello stabiliscono perché si sta facendo 
%il sistema e quali siano i suoi vantaggi. 
\begin{itemize}
    \item Creazione di un gioco di genere Roguelike
    \item Possibilità di gioco su browser
\end{itemize}

\subsection{Requisiti utente}
%Quando un utente usa il sistema, cosa vuole fare e cosa si aspetta?
%Raccolgono le aspettative per un utente.
%Un modo tipico per documentare i requisiti utente sono le user stories.

Matteo Brocca in questo progetto ricopre il ruolo di esperto di dominio e committente. 
E' un appassionato di giochi Roguelike ma essendo troppo bravo li ha finiti tutti. 
Da qui l'idea di crearne uno nuovo per lui. 

\begin{enumerate}
    \item L'utente potrà 
    \begin{enumerate}
    \item avviare una nuova partita da un menu contestuale
    \item controllare il proprio personaggio all'interno della stanza per
    \begin{enumerate}
        \item spostarsi e cambiare direzione
        \item sparare ai nemici
        \item raccogliere elementi utili all'aumento delle sue caratteristiche
        \item spostarsi da una stanza all'altra attraverso delle porte
        \item visualizzare in modo continuo le sue statistiche e i suoi punti vita
    \end{enumerate}
    \end{enumerate}
    \item L'utente dovrà riuscire a 
    \begin{enumerate}
    \item distinguere nemici, oggetti e elementi di disturbo
    \item capire in che direzione si sta muovendo
    \item capire dove sta sparando
    \item capire di aver colpito un nemico
    \item capire di aver raccolto un oggetto
    \item capire qual'è la stanza del boss
    \item capire che sta fronteggiando un boss
    \item capire di aver vinto o perso una partita
    \end{enumerate}
\end{enumerate}

\subsection{Requisiti funzionali}
%Funzionali: statement dettagliati delle funzionalità del sistema indicate
%in modo chiaro (organizzarli permette uno sviluppo del progetto rigoroso)
\begin{enumerate}
    \item Menù di gioco
    \begin{enumerate}
        \item presenza di un tasto per avviare una nuova partita
    \end{enumerate}
    \item Generazione di una mappa di gioco 2D in maniera casuale
    \begin{enumerate}
        \item Una mappa è formata da più stanze quadrate
        \item Ogni stanza è fisicamente adiacente ad almeno un'altra stanza
        \item Tra stanze adiacenti è presente una porta che le collega
        \item Una stanza è di diverse tipologie
            \begin{enumerate}
                \item Iniziale
                    \begin{enumerate}
                        \item Vuota
                        \item Una sola nella mappa
                    \end{enumerate}
                \item Oggetto
                    \begin{enumerate}
                        \item Contiene un singolo oggetto scelto randomicamente
                    \end{enumerate}
                \item Combattimento
                    \begin{enumerate}
                        \item Il contenuto è generato randomicamente
                        \item Contiene alcuni nemici di una singola tipologia
                        \item Contiene alcuni elementi bloccanti
                        \item Le porte si chiudono quando il giocatore entra nella stanza
                        \item Le porte si aprono quando il giocatore ha eliminato tutti i nemici
                    \end{enumerate}
                \item Boss
                    \begin{enumerate}
                        \item Contiene solamente il boss
                        \item Una sola nella mappa
                        \item Adiacente ad una ed una sola altra stanza
                        \item Riconoscibile grazie ad una porta particolare 
                    \end{enumerate}
            \end{enumerate}
        \item A video è visibile solo la stanza dove è presente il personaggio
        
    \end{enumerate}
    
    \item Personaggio controllabile con caratteristiche e punti vita
    \begin{enumerate}
        \item Il personaggio è caratterizzato da
            \begin{enumerate}
                \item Punti vita
                \item Velocità di movimento
                \item Danno
                \item Rate di fuoco
            \end{enumerate}
        \item Il personaggio è controllato dall'utente con la tastiera
        \begin{enumerate}
            \item L'utente può muovere il personaggio nelle quattro direzioni principali (sopra, sotto, destra, sinistra)
            \item L'utente può eseguire uno o più "shot" verso una delle quattro direzioni principali (sopra, sotto, destra, sinistra)
            \item Il personaggio è vincolato a stare nei limiti della stanza e può uscire solo dalle porte
        \end{enumerate}
        \item Il personaggio infligge danno ai nemici quando uno "shot" colpisce un nemico
        \item Il personaggio può raccogliere oggetti che modificano le sue caratteristiche
    \end{enumerate}
    
    \item Nemici con diverse caratteristiche
    \begin{enumerate}
        \item Un nemico può essere di diverse tipologie
             \begin{enumerate}
                \item A (nome del nemico)
                \item B
                \item C
            \end{enumerate}
        \item Ogni tipologia è caratterizzata da
            \begin{enumerate}
                \item Punti vita
                \item Velocità di movimento
                \item Danno
                \item Modalità di movimento
                 \begin{enumerate}
                    \item Casuale all'interno di una stanza
                    \item In direzione del giocatore
                \end{enumerate}
            \end{enumerate}
        \item Un nemico infligge danno al personaggio quando lo tocca
        \item Un nemico si muove solo all'interno di una stanza
    \end{enumerate}
    
    \item Oggetti con diversi comportamenti
    \begin{enumerate}
        \item Un oggetto può essere di diverse tipologie
             \begin{enumerate}
                \item A (nome dell'oggetto)
                \item B
                \item C
            \end{enumerate}
        \item Un oggetto in base alla sua tipologia varia alcune caratteristiche del giocatore 
    \end{enumerate}
\end{enumerate}


\subsection{Requisiti non funzionali}
%Non funzionali: statement dettagliati che raccolgono la qualità del 
%comportamento o sui vincoli del sistema

\begin{enumerate}
    \item Fluidità di gioco
    \begin{enumerate}
        \item Il gioco non deve presentare lag o inestetismi marcati che rendano la user experience non ottimale
    \end{enumerate}
    \item Bilanciare numero e caratteristiche dei nemici per ottenere un livello di difficoltà di gioco medio
    \begin{enumerate}
        \item Il gioco non deve essere troppo facile 
        \begin{enumerate}
            \item Presenza di pochi nemici all'interno di una stanza
            \item Presenza di oggetti che potenzino le caratteristiche del giocatore rendendo il conflitto giocatore-nemici impari
        \end{enumerate}
        \item Il gioco non deve essere troppo difficile
        \begin{enumerate}
            \item Presenza di troppi nemici all'interno di una stanza
            \item Presenza di oggetti che non potenzino o riducano le caratteristiche del giocatore rendendo il conflitto giocatore-nemici impari
        \end{enumerate}
    \end{enumerate}
\end{enumerate}


\subsection{Requisiti di implementazione}
%Implementazione: requirement posti agli sviluppatori. Danno indicazioni 
%sul lavoro da fare. Possono comprendere: tipi di strumenti utilizzati, 
%modi di sviluppo, documentazione fornita.

\begin{enumerate}
\item Utilizzo del game engine Indigo
\item Implementazione mediante Prolog del movimento dei nemici
\item Testing mediante ScalaTest
\item Sviluppo del software in modo da poter aggiungere nuovi componenti o estensioni a quelli gia esistenti in modo semplice
\begin{enumerate}
    \item Aggiunta di diversi personaggi con cui giocatore
    \item Aggiunta di nuovi nemici e boss
    \item Aggiunta di nuovi elementi bloccanti
    \item Aggiunta di nuovi oggetti con diverse caratteristiche
\end{enumerate}
\item Rilascio del gioco su server tramite GitHub Actions
\end{enumerate}